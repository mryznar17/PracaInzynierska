\chapter{Podstawy teoretyczne}
\label{cha:podstawyTeoretyczne}
Rozdział ten przedstawia podstawy teoretyczne projektu. W pierwszym porozdziale znajduje się krótkie wprowadzenie do architektóry SOA, na której oparty jest serwis. W następnym opisany jest jezyk implementacji. Trzeci podrozdział prezentuje formę przechowywania danych w serwisie. Ostatni punkt opisuje zaznajamia nas z przenaczeniem serwerów aplikacyjnych i konkretnym zastosowaniu w implementowanym projekcie.

%---------------------------------------------------------------------------

\section{SOA}
\label{sec:soa}

Rozwinięciem skrótu SOA jest \textit{Service Oriented Architecture} czyli architektura zorientowana na usługi. Według książki \textit{SOA Design Principles for Dummies} SOA składa się z trzech podstawowych aspektów:
\begin{itemize}
	\item \textbf{Usługa} - czyli powtarzalne zadanie biznesowe np. Dodawanie plików na serwer lub tworzenie nowego konta.
	\item \textbf{Orientacja na usługi} - Sposób postrzegania produktu jako zbiór usług połączonych w całość.
	\item \textbf{SOA} - Podejście architektoniczne? bazujące na zasadach orientacji na usługi.
\end{itemize}
Architektura zorientowana na usługi posiada następujące właściwości:
\begin{itemize}
	\item \textbf{Luźne powiązanie usług}
	\item \textbf{Abstrakcyjne usługi}
	\item \textbf{Autonomiczność usług}
	\item \textbf{Możliwość wielokrotnego użycia usług}
	\item \textbf{Bezstanowość usług}
	\item \textbf{Dzielenie usług na komponenty}
\end{itemize}
\cite{SOA13}
SOA jest architekturą wspierającą modularność więc produkty pisane w tej architekturze łatwo jest rozwijać poprzez dodawanie nowych usług. \newline
- a tutaj moze ze moze na projekt studencki to nie jest najlepsze rozwiazanie, ale jezeli mialo by to sie rozwinac to to jest super ekstra pomysl i dlaczego tak jest! (Tutaj zapewne wpomiec, ze soa jest modularne i bez problemu mozna dodawac kolejne moduly).\newline

\section{Java}
\label{sec:java}

Java została stworzona na początku lat dziewięćdziesiątych przez grupę inżynierów z firmy Sun zwaną "Green Team". Został stworzony do pracy w rozwijającym się środowisku internetu.
- Na poczatek o samym jezyku, troche historii, ze na rynku jeden z najczesciej uzywanych itd.\newline
- 
- No i tutaj napisac, ze Java to super pomysl bo wsparcie dla soa poprzez platforme JEE. \newline
- Nastepnie opisac sama platforme - to tez troche zajmie \newline
- Jak juz bede przy tej platformie to przyjebac ze dwie podsekcje o jsf i hibernate \newline

%---------------------------------------------------------------------------

\section{Przechowywanie danych}
\label{sec:bazadanych}

- Tutaj będa informacje o bazie danych \newline
- najpierw moze jakie sa opcje przechowywania danych \newline
- potem dlaczego akurat baza danych \newline
- no i na koncu dlaczego postgres \newline
%---------------------------------------------------------------------------
\section{Serwer Aplikacyjny}
\label{sec:serweraplikacyjny}

- Na samym poczatku napisac za co odpowiada serwer aplikacyjny \newline
- Od kiedy używane i dlaczego sa przydatne \newline
- Najbardziej znane wymienic i opisac zdawkowo \newline
- Opisac swoj wybor - dlaczego JBoss \newline
- JBoss jest darmowy \newline
- wspiera w pelni JEE \newline
- jest szybki \newline