\chapter{Implementacja}
\label{cha:implementacja}
Rozdział ten opisuje najważniejsze szczegóły implementacyjne projektu. W pierwszym podrozdziale znajduje się krótki opis narzędzi użytych do implementacji oraz sposób w jaki ich użyto. W drugim znajduje się opis podprojektów, a w trzecim metody zabezpieczeń systemu.
\section{Użyte narzędzia}
Aplikacja została zaimplementowana w środowisku programowania \textbf{Eclipse IDE for Java EE Developers}, ponieważ posiada wszystkie niezbędne narzędzia do tworzenia aplikacji JEE. Innym plusem użytego środowiska jest także dodatkowy plugin JBossTools odpowiadający za zarządzanie serwerem aplikacyjnym JBoss.

Dodatkowym bardzo przydatnym narzędziem użytym do implementacji projektu jest \textbf{Apache Maven}, który jest odpowiedzialny za budowanie aplikacji. Tworzenie aplikacji przy użyciu narzędzia maven opiera się na pliku pom.xml (POM – ang. Project Object Model). Jest on miejscem, w którym znajdują się wszystkie najważniejsze informację definiujące projekt, jego strukturę, to w jaki sposób będzie budowany oraz zależności \cite{MV01}. Dużą zaletą Mavena jest łątwy sposób dodawania potrzebnych bibliotek do projektu, co odbywa się za pomocą przyłączania kolejnych zależności do pliku pom.xml, które są automatycznie ściągane z repozytorium mavena. Przykład dodania biblioteki JSF w projekcie:
\begin{lstlisting}
      <dependency>
         <groupId>org.jboss.spec.javax.faces</groupId>
         <artifactId>jboss-jsf-api_2.1_spec</artifactId>
         <scope>provided</scope>
      </dependency>
\end{lstlisting}
Maven posiada archetypy czyli dodatkowe narzędzie do generowania szablonów projektów. Przy tworzeniu opisywanej aplikacji skorzystano z tej właśnie funkcjonalności. Szablon tworzy się za pomocą wywołania komendy:
\begin{lstlisting}
mvn archetype:generate 
\end{lstlisting}
Następnie z pośród tysięcy dostępnych szablonów trzeba wybrać ten, który nas interesuje. W przypadku tego projektu jest to archetyp który buduje szablon aplikacjii JavaEE kompatybilny z serwerem aplikacyjnym JBoss AS 7.1:
\begin{lstlisting}[breaklines=true]
remote -> org.jboss.spec.archetypes:jboss-javaee6-webapp-ear-archetype (An archetype that generates a starter Java EE 6 webapp project for JBoss AS 7.1 (by default) or EAP 6 (if the "enterprise" property is true). The project is an EAR, with an EJB-JAR and WAR)
\end{lstlisting}

Po wybraniu archetypu następuje definicja nazwy projektu oraz nazwy domyślnego pakietu. Wybrany pakiet dzieli projekt na trzy podprojekty: 
\begin{itemize}
	\item EJB - Odpowiedzialny za połączenie z bazą danych. Tutaj znajdują się klasy reprezentujące tabele w bazie danych oraz klasy Bean, które są odpowiedzialne za pobieranie i wstawianie danych z bazy. Podprojekt EJB eksportowany jest w postaci plików JAR (ang. Java archive),
	\item WEB - W tym podprojekcie znajdują się wszystkie strony html tworzące interfejs użytkownika, oraz klasy Managed Bean, które są odpowiedzialne za połączenie ze stronami html. Podczas budowy aplikacji podprojekt WEB zapisywany jest jako WAR (ang. Web archiwe),
	\item EAR - Aplikacja eksportowana jest w postaci plików EAR (enterprise archiwe), które zawierają wszystkie podprojekty implementowanej aplikacji \cite{ECL01}.
%	http://help.eclipse.org/mars/index.jsp?topic=%2Forg.eclipse.jst.j2ee.doc.user%2Ftopics%2Fcjearproj.html).
\end{itemize}
Zbudowany projekt EAR jest lokowany na serwerze aplikacyjnym JBoss.\newline

\section{Oprogramowanie}
Projekt jak to zostało opisane w poprzednim podrozdziale jest podzielony na trzy podprojekty, z których implementacja znajduje się tylko w częściach WEB i EJB. W tym podrozdziale przybliżona zostanie ich zawartość.
\subsection{EJB}
\label{subsec:ejb}
W opisywanym projekcie znajdują się klasy, które zostały wygenerowane dzięki narzędziu JPA Tools (ang. Java Persistance API) z tabel bazy danych opisanych w punkcie \ref{sec:bd}. Przykładowa wygenerowana klasa reprezentująca użytkownika:
\begin{lstlisting}
package com.inz.model;

import java.io.Serializable;
import javax.persistence.*;
import java.util.List;

/**
 * The persistent class for the users database table.
 * 
 */
@Entity
@Table(name="users")
@NamedQueries({
	@NamedQuery(name="user.findAll", query="SELECT u FROM User u"),
	@NamedQuery(name="user.byId", query="select u from User as u where u.userId=:userId"),
	@NamedQuery(name="user.byLogin", query="select u from User as u where u.login=:login"),
})
public class User implements Serializable {
	private static final long serialVersionUID = 1L;

	@Id
	@Column(name="user_id", columnDefinition = "serial")
	private Integer userId;
	private String email;
	private String login;
	private String name;
	private String password;
	private String surname;

	//bi-directional many-to-one association to Calendar
	@OneToMany(mappedBy="user")
	private List<Calendar> calendars;

	//bi-directional many-to-one association to File
	@OneToMany(mappedBy="user")
	private List<File> files;

	//bi-directional many-to-one association to SharedFile
	@OneToMany(mappedBy="user")
	private List<SharedFile> sharedFiles;

	//bi-directional many-to-one association to Role
	@ManyToOne
	@JoinColumn(name="role_id")
	private Role role;

	public User() {
	}

	/* Metody get i set dla wszystkich atrubutów */
}
\end{lstlisting}

W powyższym przykłądzie ominięte zostały metody get i set odpowiadające za pobieranie i ustawianie atrybutów klasy. Właściwości tabeli w wygenerowanej klasie są zachowane dzięki adnotacjom (@Adnotacja), gdzie @Id definiuje klucz główny tabeli, @OneToMany i @ManyToOne relacje zachodzące między tabelami i @Column podający dodatkowe właściwości kolumny, które nie mogą być przedstawione przy pomocy zmiennej w języku Java. Ponadto ponad definicją klasy znajdują się adnotacje @Entity, która oznacza klasę jako reprezentującą tabelę z bazy danych, @Table pozwalająca ustawić nazwę tabeli do której się odwołuje oraz @NamedQueries, która pozwala na zdefiniowanie w prosty sposób zapytań do bazy danych. Dzięki frameworkowi JPA i wygenerowanym przez niego klasom nie trzeba się odwoływać do tabel z bazy, lecz do powstałych klas.

Aby ułatwić innym komponentom dostęp do bazy danych skorzystano z wzorca projektowego DAO (ang. Data Access Object), który tworzy jednolity interfejs odpowiedzialny za połączenie z bazą danych. DAO sprawia, że warstwa odpowiadająca za logike aplikacji nie musi znać szczegółów implementacyjnych udostępnionego interfejsu co uniezależna tą warstwę od sposobu dostępu do bazy danych. Poniżej zostanie przedstawiony przykładowy interfejs i jego implementacja odpowiedzialne za zarządzanie tabelą użytkownika.\\
Interfejs:
\begin{lstlisting}[breaklines=true]
package com.inz.dao;

import java.util.List;
import com.inz.model.User;

public interface UserDao {
	List<User> getUsers();
	User getUser(int id);
	User getUser(String login);
	void updateUser(User user);
	void deleteUser(int id);
	void deleteUser(String login);
	void addUser(String login, String name, String surname, String email, String password);
	void addUser(User user);
}
\end{lstlisting}
Implementacja:
\begin{lstlisting}[breaklines=true]
package com.inz.dao.impl;

import java.util.LinkedList;
import java.util.List;

import javax.ejb.Local;
import javax.ejb.Stateless;
import javax.ejb.TransactionAttribute;
import javax.ejb.TransactionAttributeType;
import javax.persistence.EntityManager;
import javax.persistence.PersistenceContext;

import com.inz.dao.UserDao;
import com.inz.model.User;

@Stateless
@Local(UserDao.class)
@TransactionAttribute(TransactionAttributeType.REQUIRED)
public class UserDaoImpl implements UserDao {

	@PersistenceContext(name="primary")
	private EntityManager em;
	
	@Override
	public List<User> getUsers() {
		return em.createNamedQuery("user.findAll").getResultList();
	}

	@Override
	public User getUser(int id) {
		return (User) em.createNamedQuery("user.byId")
			.setParameter("userId", id).getSingleResult();
	}

	@Override
	public User getUser(String login) {
		return (User) em.createNamedQuery("user.byLogin")
			.setParameter("login", login).getSingleResult();
	}

	@Override
	public void updateUser(User user) {
		em.merge(user);
	}

	@Override
	public void deleteUser(int id) {
		User u = getUser(id);
		em.remove(u);
	}

	@Override
	public void deleteUser(String login) {
		User u = getUser(login);
		em.remove(u);
	}

	@Override
	public void addUser(String login, String name, String surname, String email,
			String password) {
		User u = new User();
		u.setLogin(login);
		u.setName(name);
		u.setSurname(surname);
		u.setEmail(email);
		u.setPassword(password);
		u.setUserId(generateId());
		em.persist(u);
	}

	@Override
	public void addUser(User user) {
		if(user.getUserId().equals(null)){
			user.setUserId(generateId());
		}
		em.persist(user);
	}
	
	private int generateId(){
		List<User> users = getUsers();
		int id=0;
		List<Integer> ints = new LinkedList<Integer>();
		for(User usr:users){
			ints.add(usr.getUserId());
		}
		int i=1;
		while(true){
			if(ints.contains(i)) i++;
			else{
				id=i;
				break;
			}
		}
		return id;
	}
}
\end{lstlisting}

Najważniejszą zmienną odpowiedzialną za wykonywanie operacji na bazie danych jest instancja klasy EntityManager, która jest "wstrzyknięta" do klasy za pomocą adnotacji @PersistenceContext. Adnotacja @Stateless oznacza, że wstrzykniwany Bean (komponent wielokrotnego użytku w różnych miejscach aplikacji) będzie bezstanowy. Komponent ten jak sama nazwa wskazuje nie posiada żadnego stanu więc wszystkie takie komponenty wywoływane przez różnych użytkowników są identyczne. Adnotacja @Local mówi o tym, że implementowany interfejs znajduje się w tym samym środowisku. Adnotacja @TransactionAttribute(TransactionAttributeType.REQUIRED) wymusza na każdej metodzie, aby była wykonana w ramach tranzakcji już działającej, a jeżeli żadna nie jest rozpoczęta to musi być utworzona nowa. Stworzony bean wstrzykuje się w następujący sposób:
\begin{lstlisting}
@EJB private UserDao dao;
\end{lstlisting}

\subsection{WEB}
Użytkownik korzysta z aplikacją za pomocą przeglądarki internetowej, więc interfejsem są strony www. W implementowanej aplikacji są to pliki z rozszerzeniem xhtml. Głównym frameworkiem pomocnym przy tworzeniu tego interfejsu jest JSF, który udostępnia dodatkowe tagi pomocne przy tworzeniu stron oraz klasy Managed Bean dzięki którym można się połączyć z tymi stronami.\\
Przykładowy formularz odpowiedzialny za rejestracje nowego użytkownika:
\begin{lstlisting}[breaklines=true]
<?xml version="1.0" encoding="UTF-8" ?>
<!DOCTYPE html PUBLIC "-//W3C//DTD XHTML 1.0 Transitional//EN" "http://www.w3.org/TR/xhtml1/DTD/xhtml1-transitional.dtd">
<html 	xmlns="http://www.w3.org/1999/xhtml" 
		xmlns:h="http://java.sun.com/jsf/html"
		xmlns:f="http://java.sun.com/jsf/core"> 
<h:head>
<meta http-equiv="Content-Type" content="text/html; charset=UTF-8" />
<title>Rejestracja</title>
</h:head> 
<h:body>  
	Zarejestruj się: 
	<h:form>
	<br />
	   Login: <h:inputText id="login" value="#{userMan.login}" 
	   			required="true"
	   			validator="#{userMan.validateLogin}"
	   			requiredMessage="Pole Login jest wymagane."/> 
	   			<h:message for="login" style="color:red" /> <br />
	   Hasło: <h:inputSecret id="password" value="#{userMan.password}" 
					required="true" 
	   			validatorMessage="Błąd walidacji: Hasło nie może zawierać znaków specjalnych oraz musi mieć długość od 5 do 30 znaków."
	   			requiredMessage="Pole Hasło jest wymagane.">  
	   			<f:validateRegex
					pattern="[A-Za-z0-9]*{5,30}$" />
			    </h:inputSecret>
			    <h:message for="password" style="color:red" /> <br />
		Imię: <h:inputText id="name" value="#{userMan.name}" 
				required="true"  
				requiredMessage="Pole Imię jest wymagane."/> 
				<h:message for="name" style="color:red" /> <br />
	Nazwisko: <h:inputText id="surname" value="#{userMan.surname}" 
				required="true" 
				requiredMessage="Pole Nazwisko jest wymagane."/> 
				<h:message for="surname" style="color:red" /> <br />
	   Email: <h:inputText id="email" value="#{userMan.email}" 
				required="true" 
				validatorMessage="Błąd walidacji: Email musi być postaci użytkownik@domena."
				requiredMessage="Pole Email jest wymagane."> 
				<f:validateRegex
					pattern="^[_A-Za-z0-9-\+]+(\.[_A-Za-z0-9-]+)*@[A-Za-z0-9-]+(\.[A-Za-z0-9]+)*(\.[A-Za-z]{2,})$" />
			  </h:inputText>
			  <h:message for="email" style="color:red" /> <br />
		<h:commandButton value="Dodaj mnie" action="#{userMan.addUser()}" />
	</h:form>
</h:body>
</html>
\end{lstlisting}
Klasa Managed Bean odpowiedzialna za obsługę powyższego formularza:
\begin{lstlisting}[breaklines=true]
package com.inz.jsf;

import java.util.List;
import javax.ejb.EJB;
import javax.faces.application.FacesMessage;
import javax.faces.bean.ManagedBean;
import javax.faces.bean.SessionScoped;
import javax.faces.component.UIComponent;
import javax.faces.context.FacesContext;
import javax.faces.validator.ValidatorException;
import com.inz.dao.UserDao;
import com.inz.model.User;

@ManagedBean(name = "userMan", eager = true)
@SessionScoped
public class UserManager {
	@EJB private UserDao dao;
	private String login;
	private String name;
	private String surname;
	private String password;
	private String email;
	
	public void ManagedBean(){};
	
/* Metody get i set */

	public String addUser(){
		dao.addUser(login, name, surname, email, password);
		return "index";
	}
	
	public void validateLogin(FacesContext context, UIComponent component, Object value) throws ValidatorException {
		  String login = (String) value;
		  String message = "Błąd walidacji: ";
		  boolean err = false;
		  if ( login.length() > 30 || login.length() < 5 ) {
			  message=message+"Login musi mieś długość od 5 do 30 znaków.";
			  err=true;
		  } else if(validateLoginHelper(login)){
			  message=message+"Użytkownik o podanym loginie znajduje się już w bazie danych.";
			  err=true;
		  }  
		  if(err){
			  FacesMessage msg = new FacesMessage(message);
		      throw new ValidatorException(msg);
		  }
		}
	
	private boolean validateLoginHelper(String login){
		List<User> users = dao.getUsers();
		for(User u:users){
			if(u.getLogin().equals(login)) return true;
		}
		return false;
	}
}
\end{lstlisting}

ManagedBean to klasa z frameworku JSF, która jest dostępna ze stron używających tagów JSF. Definiowana jest za pomocą adnotacji @ManagedBean. Adnotacja @SessionScoped definiuje, że Bean żyje tak długo jak sesja HTTP. 

Aby użytkownik nie mógł wprowadzać nieprawidłowych danych przy tworzeniu formularza zastosowano mechanizm walidacji wprowadzanych danych. Użytkownik nie może zostawić pola nieuzupełnionego (required="true"), a ponadto do pola login, hasło i email są sprawdzane pod względem wprowadzanego tekstu. W polach email i hasło sprawdzany jest warunek długości oraz treści wprowadzanego tekstu za pomocą wyrażenia regularnego na stronie JSF (f:validateRegex pattern="warunek"). Pole login walidowane jest za pomocą MangedBeana pod względem długości tekstu (od 5 do 30 znaków) oraz unikalności loginu w systemie. Jeżeli któryś z wyżej opisanych warunków nie zostanie spełniony to po prawej stronie pola pojawi się odpowiedni komunikat w kolorze czerwonym.

\section{Zabezpieczenia}
Wszystkie usługi będą udostępniane tylko zalogowanym użytkownikom. Możliwość zarządzania plikami oraz kalendarzem wszystkich użytkowników posiada tylko administrator, co jest autoryzowane za pomocą bazy danych oraz podsystemu bezpieczeństwa serwera aplikacyjnego JBoss. 

W celu stworzenia logowania do systemu skorzystano z pomocy modułu logowania \textit{Database} dostarczanego przez serwer aplikacyjny JBoss. Aby zrealizować autoryzację w pierwszej kolejności trzeba dodać odpowiednie elementy do pliku konfiguracyjnego serwera JBoss (standalone.xml):
\begin{lstlisting}[breaklines=true]
<security-domain name="postgresql-domain" cache-type="default">
	<authentication>
		<login-module code="Database" flag="required">
			<module-option name="dsJndiName" value="java:jboss/datasources/StudentServisDataSource"/>
			<module-option name="principalsQuery" value="select password from users where login=?"/>
			<module-option name="rolesQuery" value="select role_name, 'Roles' as Roles from role r join users u on (r.role_id=u.role_id) where u.login=?"/>
		</login-module>
	</authentication>
</security-domain>
\end{lstlisting}
Powyższy wpis definiuje moduł logowania korzystający z weryfikacji za pomocą bazy danych zdefiniowanej w źródle danych \textit{StudentServisDataSource}. Następnie przygotowany jest specjalny formularz logowania, po czym w pliku xml między znacznikami <security-constraint> definiowane są strony, do których dostęp ma tylko zalogowany użytkownik. Dodatkowo w pliku web.xml definiowane są strony logowania do których niezalogowany użytkownik jest przekierowywany gdy chce uzyskać dostęp do stron objętych ochroną:
\begin{lstlisting}[breaklines=true]
		<form-login-config>
			<form-login-page>/login.xhtml</form-login-page>
			<form-error-page>/login_error.xhtml</form-error-page>
		</form-login-config>
\end{lstlisting}
Aby przydzielić niektóre funkcje systemu tylko administratorowi nad ManagedBeanem wstawiane są następujące adnotacje:
\begin{lstlisting}[breaklines=true]
@RolesAllowed("admin")
@SecurityDomain("postgresql-domain") 
\end{lstlisting} 
Gdzie adnotacja @SecurityDomain definiuje baze danych na podstawie której użytkownik rola użytkownika jest weryfikowana.
