\chapter{Podsumowanie}
\label{cha:podsumowanie}

Celem pracy była implementacja serwisu internetowego przeznaczonego dla studnetów. Główną funkcjonalnością sytemu miała być możliwość dodawania plików do systemu oraz udostępnianie tych plików znajomym użytkownikom. Ponadto każdy użytkownik miał mieć prywatny kalendarz w celu dodawania ważnych wydarzeń. Ostatnią zaplanowaną funkcjonalnością miała być strona powiadomień, która miała by przypominać o zaplanowanych wydarzeniach, udostępnionych plikach oraz o wydarzeniach ze strony uczelni.

Cel pracy został osiągnięty, w wyniku implementacji powstał serwis internetowy zaipmementowany w architekturze SOA. Serwis posiada zaplanowaną funkcjonalność. Przechowywanie plików zostało zrealizowane za pomocą zapisywania plików w postaci binarnej w bazie danych. Interfejs użytkownika są strony internetowe utworzone za pomocą frameworku JSF. Niestety nie udało się zrealizować funkcjonalności pobierania aktualności z portalu Facebook z oficjalnych grup założonych przez uczelnie do której uczęszcza użytkownik.

W przypadku kontunuowania pracy przy projekcie poprawiony by został interfejs użytkownika za pomocą skorzystania z odpowiednich bibliotek lub narzędzi pomagających w jego tworzeniu. Ponadto gdyby serwis miałby być używany na szeroką skalę trzeba by wziąć pod uwagę kwestię lepszego zabezpieczenia danych użytkownika oraz lepszy system przechowywania plików.

Projekt dzięki architekturze SOA można rozwijać poprzez dodawanie kolejnych modułów dodających nową funkcjonalność. W przypadku rozwijania aplikacji dobrym pomysłem było by dodanie integracji w portalami społecznościowymi w celu zwiększenia zainteresowania serwisem oraz ułatwienia do niego dostępu. Kolejną funkcjonalnością mogącą przynieść serwisowi większe zainteresowanie było by przygotowanie aplikacji mobilnej na system Adroid.

%W podsumowaniu pracy należy zebrać wnioski z jej realizacji. Odpowiedzieć na pytania: czy
%cel pracy został osiągnięty i w jakim stopniu. Co inaczej byłoby realizowane, gdyby autor od
%nowa zaczął tę pracę? Można tu podać trochę ciekawostek z jej realizacji. Kto i jaki pożytek
%może mieć z tej pracy? Czy warto kontynuować tą pracę i w jaki sposób? Jakie nowe
%problemy zostały zidentyfikowane i które z nich mogę być przedmiotem kolejnych prac
%dyplomowych? Podsumowanie nie ma na celu wykazywać, że stworzony produkt jest idealny
%i bezbłędny, lecz to, że autor jest rzetelnym i kompetentnym projektantem i analitykiem. Do
%błędów popełnionych w pracy należy się uczciwie przyznać. \newline