\chapter{Projekt systemu}
\label{cha:projektSystemu}

\section{Opis systemu}
Tutaj nad podtytułami troszkę pomyśleć.
Pewnie jak wszedzie krótki opis co w ktorym opisane, po co ten rozdzial.
\subsection{Cel projektu}
Tutaj z grubsza napisze co jest celem projektu. Czyli implementacja serwisu internetowego, którego zadaniem jest wspomaganie wymiany informacji i plików pomiędzy studnetami.

\subsection{Udziałowcy i użytkownicy}
Decelowy użytkownik to student potrzebujący wymieniać pliki potrzebne na zajęcia. Aczkolwiek nie jest to warunek wymagany, by użytkownik był studnetem.\newline
- tutaj popisać trochę o innych aktorach - administrator \newline
- co jest głównym celem użytkownika - komunikacja z innymi użytkownikami bez fizycznego spotkania? \newline

\subsection{Granice systemu}
Tutaj coś ciekawego trzeba wymyślić - popatrzeć na projekt do szweda.

\subsection{Diagramy aktywności}
- tu pewnie trtzeba będzie zmienić tytuł tego podtytułu \newline
- Diagramy aktywności przedstawiające
artefakty świata rzeczywistego - –
dodatkowy opis  \newline
Co do powyższego - trzeba wypisać podstawowe funkcjonalności czyli: \newline
	-- zarządzanie plikami \newline
	-- zarządzanie kalendarzem \newline
	-- obsługa powiadomien \newline
	-- zobaczym co jeszcze \newline
	
\section{Specyfikacja wymagań}
Po co? na co?
\subsection{Wymagania oprogramowania}
Tutaj wymanagania funkcjonalne i niefunkcjonalne z pracy wykonanej do rogusa.
\subsection{Przypadki użycia}
Napisać po co przypadki. Diagram X.x prezentuje możliwe interakcje użytkownika z systemem.\newline
DIAGRAM USE CASE\newline
SCENARIUSZE USE CASE \newline

\section{Projekt}
nad tym tytulem pomyslec!\newline
\subsection{Architektutra systemu}
- najpierw napisac ze soa\newline
- zastanowic sie czy tutaj opis soa czy we wstepie, ale chyba we wstepie, tutaj jedynie można wspomiec ze wczesniej opisany w punkcie X.x.\newline
- Diagram \newline
- byc moze opis

\subsection{Model bazy danych}
- tutaj jakis ladny wstep.\newline
- MODEL ERD - ostateczny \newline

\subsection{Model klas}
- rowniez jakis ladny wstep \newline
- No i diagram klas tutaj ładnie sklepać trzeba będzie \newline

\subsection{Diagramy sekwencji}
- Wstep, ze to wynika z diagramu klas, cos ładnego o diagramach sekwencji - tutaj moze sie wykładami szweda posłużyć \newline
- No i wklepac diagramy sekwencji - moze nie bedzie tak zle jak już beda diagramy klas \newline
\section {Projekt interfejsu?}
Nad tym sie zastanowic czy to robic czy dac dupie siana

