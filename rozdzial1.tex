\chapter{Wprowadzenie}
\label{cha:wprowadzenie}

W dobie współczesnej technologii i internetu, wymiana notatek z zajęć dydaktycznych oraz przekazywanie sobie wzajemnie przez studentów informacji dotyczących odwołanych zajęć bądź też informacji o kolokwiach i egzaminach, nie stanowi problemu. Najbardziej znanym sposobem na wymianę plików jest wysyłanie za pomocą wiadomości email. Często niestety bywa tak, że załączone materiały przekraczają dopuszczalny rozmiar załącznika, wtedy z pomocą przychodzą dyski internetowe, które pozwalają na przechowywanie większych plików. Najbardziej znane wśród studentów to google drive, dropbox i One drive.

Wszystkie wymienione wyżej rozwiązania są często stosowane przez studentów, aczkolwiek trudno znaleść odpowiedni serwis dedykowany bezpośrednio dla nich. Celem tej pracy jest implementacja serwisu internetowego pomagającego w zarządzaniu materiałami dydaktycznymi z zajęć. Głównym przeznaczeniem tworzonego portalu jest dodawanie plików oraz udostępnianie ich innym użytkownikom. Dodatkową planowaną  funkcjonalnością, która może pomóc studentowi w organizacji roku akademickiego jest przeznaczony dla niego kalendarz, w którym użytkownik mógłby zapisywać wszystkie ważne wydarzenia takie jak kolokwia lub egzaminy. Kolejną funkcją systemu, która ma na celu ułatwienie studentowi zarządzania kontem serwisu jest strona powiadomień na której znajdować się będą przypomnienia o zbliżających się wydarzeniach, informacje o udostępnionych dla niego plikach oraz informacje dotyczące uczelni.

Niniejsza praca opisuje proces powstawania przedstawionego serwisu. W pierwszym rozdziale czytelnik zostanie zaznajomiony z najważniejszymi pojęciami dotyczącymi opisywanego projektu. W kolejnym rozdziale znajduje się koncepcja systemu przedstawiona za pomocą dokładnie opisanych (słownie oraz za pomocą diagramów czynności oraz przypadków użycia) zaplanowanych funkcji systemu. W czwartym rozdziale za pomocą modelu bazy danych oraz modelu klas przedstawiony jest projekt systemu. Na samym końcu opisane są najważniejsze szczegóły implementacyjne projektu.