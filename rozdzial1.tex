\chapter{Wprowadzenie}
\label{cha:wprowadzenie}

\section{Przedmowa - Przerobic na ladny tekst}
- Cos o tym ze dawnej studenci musieli zadupcac kserowac notatki, spisywac od znajomych. No ale zaznaczyc to ze musieli zadupcac po to wszystko \newline
- nagle buch - internet sie stal - wszystkie wymienione wyzej problemy mozna rozwiazac nie ruszajac sie z fotela. \newline
- Gdy sie nie wie kiedy kolos - piszesz na portalu społecznościowym, forum itd. (Tutaj moze jak sie da to ladnie wstawic to przyklady portali fejs,twitter,fora itd)\newline
-A gdy potzrebujesz notatki to albo wysyla Ci ktos mailem, albo udostepnia na dysku internetowym, co tez ulatwia zdecydowanie zycie (No i tutaj opisac dyski internetowe? ze google wiedzie prym bo ma maila i google dysk, dodatkowo napisac o innych istaniejacych rozwiazaniach (dropbox, oneDrive itd) ). \newline
- Tutaj trzeba bedzie napisac o tym ze to jest wazna rzecz w życiu studenta, ze bez tego to jak bez reki. \newline
- Ze wszystkie opisane powyzej rozwiazania sa eksta itd, (ale wg mojego researchu?) nie ma rozwiazania dedykowanego stricte dla studentow tj portalu na ktorym mogli by wymieniac swobodnie pliki, zapisywac terminy w kalendarzu itd. \newline
- I tutaj BUM - ja chce cos takiego zrobic \newline
• Dlaczego autor chciał zająć się danym tematem? \newline
Przykladowe odniesienie do bibliografii \newline



\section{Cel pracy}

- Celem pracy jest stworzenie serwisu internetowego opartego na architekturze SOA.\newline
- Gdzies tu trzeba opisac co bedzie sie zawieralo w pracy, ze strona www i jaka funkcjonalnosc na tej stronie\newline
- jaki serwer \newline
- No i tuataj w sumie wspomiec, ze wybralem taki temat bo chcialem poznac technologie SOA? albo wspomiec o tym gdzies indziej\newline
- NO i tutaj po kolei opisac co robie w pracy, z grubsza z czego korzystam\newline
- Byc moze co w ktorym rozdziale tak z grubsza \newline
- To wszystko razem powinno zajac wystarczajaco miejsca \newline

• Co autor chciał osiągnąć, czego się spodziewał w efekcie realizacji pracy? \newline
• Czego czytelnik dowie się po przeczytaniu pracy? \newline


















